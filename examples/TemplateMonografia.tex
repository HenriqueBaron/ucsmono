\documentclass[12pt,oneside,english,brazil,lmodern]{ucsmonograph}
\usepackage[utf8]{inputenc}
\usepackage[T1]{fontenc}
\usepackage{graphicx}
\usepackage{pdfpages}
\usepackage{amsmath}

\titulo{Meu trabalho de conclusão de curso}
\autor{Pedro Silva}
\data{2018}
\instituicao{Universidade de Caxias do Sul}
\local{Caxias do Sul}
\preambulo{Trabalho de conclusão de curso apresentado à Universidade de Caxias do Sul como requisito parcial à obtenção de grau de Engenheiro Mecânico. Área de concentração: Projetos de Máquinas: Estática e Dinâmica Aplicada.}
\orientador{Dr. Orientador do Meu Trabalho}
\palavraschave{Minhas. Palavras. Chave}

\begin{document}
	\imprimircapa
	\imprimirfolhaderosto
	
	\imprimirfolhadeaprovacao[Empresa ABC Ltda.]{10/12/2018}{Dr. Avaliador Banca 1}{Me. Avaliador Banca 2}{Eng. Avaliador Externo}
	
	\begin{dedicatoria}
		Dedico este trabalho à minha mãe, meu pai e meu cachorro\dots
	\end{dedicatoria}

	\begin{resumo}
		\SingleSpacing
		Este é o resumo do trabalho.
		\vspace{\onelineskip}
		
		\noindent
		\textbf{Palavras-chave}: Palavras. Chave. Deste. Trabalho.
	\end{resumo}
	
	\begin{resumo}[Abstract]
		\SingleSpacing
		This is the abstract of this work.
		\vspace{\onelineskip}
		
		\noindent
		\textbf{Keywords}: Keywords. Of. This. Work.
	\end{resumo}
	
	\listoffigures*
	\cleardoublepage
		
	\listofquadros*
	\cleardoublepage
	
	\listoftables*
	\cleardoublepage
	
	\begin{siglas} % Não esquecer de ordenar alfabeticamente!
		\item[ABNT] Associação Brasileira de Normas Técnicas
		\item[UCS] Universidade de Caxias do Sul
		\item[XML] \foreignlanguage{english}{Extensible Markup Language}
	\end{siglas}

	\begin{simbolos}
		\item[\ensuremath{\alpha}] Ângulo qualquer [rad]
		\item[\ensuremath{\tau}] Período [s]
		\item[\ensuremath{\omega}] Velocidade angular [rad/s]
		\item[\ensuremath{t}] Tempo [s]
	\end{simbolos}

	\tableofcontents*
	
	\textual % Início dos elementos textuais
	
	\chapter{Introdução}
	Introdução do trabalho
	
	\section{Justificativa}
	Justifique a proposta.
	
	\section{Ambiente de desenvolvimento}
	Este trabalho foi desenvolvido na empresa A.
	
	\section{Objetivos}
	Aqui ficam os objetivos do trabalho.
	
	\subsection{Objetivos gerais}
	Objetivos gerais\dots
	
	\subsection{Objetivos específicos}
	Objetivos específicos\dots
	
	\chapter{Referencial teórico}
	Aqui ficam os fundamentos teóricos do trabalho proposto \cite{rao:2008}.
	
	\section{Método de Rayleigh-Litz}
	Conforme \citeonline{cooley:1965}, o método de Rayleigh-Litz.
	
	\section{Exemplo de tabela}
	A tabela exibida abaixo é um exemplo:
	\begin{table}[h]
		\centering
		\caption{Carga térmica das superfícies transparentes}
		\setlength{\doublerulesep}{\arrayrulewidth}
		\begin{tabular}{c|c|c|c|c|c|c}
			\hline
			\hline
			& Quant. & Espessura (mm) & Área ($m^2$) & FGCI & CS & $\dot{Q}_{sg}$ (W) \\ \hline
			Janela lateral  & 1 & 5 & 0,7230 & 672 & 1 & 485,856  \\ 
			esquerda (sul)  &   &   &        &     &   &          \\ 
			Janela lateral  & 1 & 5 & 0,7230 & 145 & 1 &  104,835 \\
			direita (norte) &   &   &        &     &   &          \\
			Janela superior & 1 & 5 & 0,2356 & 672 & 1 & 158,323  \\
			(oeste)         &   &   &        &     &   &          \\
			Porta dianteira & 1 & 6 & 0,7679 & 672 & 1 & 516,028  \\
			(oeste)         &   &   &        &     &   &          \\
			Janela traseira & 1 & 5 & 0,3398 & 142 & 1 &  48,252  \\
			(leste)         &   &   &        &     &   &          \\ \hline 
			&   &   &        &     &   &  1.313,3 \\ \hline \hline 
		\end{tabular}
		\fonte{o autor (\thedate)}
	\end{table}
	
	\postextual
	
	\bibliography{Bibliografia}
	
\end{document}